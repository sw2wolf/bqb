%pdflatex -shell-escape *.tex
\documentclass{standalone}
\usepackage{amsmath}
\usepackage{amsfonts} 
\usepackage{amssymb}
\usepackage{graphicx}
\graphicspath{ {images/} }
\usepackage{xcolor}
%
\pagestyle{empty}
\pagecolor{white}
%
\everymath{\displaystyle}
%
\usepackage{CJKutf8}
\AtBeginDvi{\input{zhwinfonts}}
%\input{zhwinfonts}
\newcommand{\zh}[1]{\begin{CJK*}{UTF8}{zhsong}#1\end{CJK*}}
\newcommand{\YBao}{\color{red}\zh{源宝爱数学}}
\newcommand{\BT}[1]{\color{red}\zh{#1}\color{black}}
\newcommand{\fTxT}[1]{\text{\zh{#1}}}
%

\immediate\write18{pdflatex junior20.tex}
\immediate\write18{convert -density 150 -adaptive-resize 480x480 junior20.pdf junior20.jpg}
%
\begin{document}\begin{minipage}[b][14cm][t]{\textwidth}
\begin{center}\large\BT{高二下学期末数学考试20题}\end{center}
\color{black}\begin{large}
\zh{已知等差数列$\{a_n\}$满足:$a_3=7,a_5+a_7=26.\{a_n\}$的前n项和为$S_n.$}\\
\zh{(1)求$a_n$及$S_n$};\\
\zh{(2)令$b_n=\frac{1}{a_n^2-1}\hspace{5pt}(n \in N^*)$求数列$\{b_n\}$的前n项和$T_n$}
\end{large}\\[10pt]
%%%
\begin{large}(1)\zh{解:}
\zh{设$\{a_n\}$公差为d}\\
$\Longrightarrow \begin{cases}
  a_3=a_1+2d=7 \\
  a_5+a_7=2a_1+10d=26
\end{cases}$ $\Rightarrow a_1=3,d=2$\\
$\therefore a_n=3+2(n-1)=2n+1,S_n=3n+\frac{n(n-1}{2} \times 2=n^2+2n$
\end{large}\\[10pt]
%%% 
\begin{large}(2)\zh{解:}
  $\because b_n=\frac{1}{(a_n+1)(a_n-1)}=\frac{1}{4}(\frac{1}{n}-\frac{1}{n+1})$\\
  $\therefore T_n=\frac{1}{4}(1-\frac{1}{2}+\frac{1}{2}-\frac{1}{3} \dots \frac{1}{n}-\frac{1}{n+1})=\frac{1}{4}(1-\frac{1}{n+1})=\frac{1}{4} \times \frac{n}{n+1}$
\end{large}
\end{minipage}\end{document}
