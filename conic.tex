%pdflatex -shell-escape *.tex
\documentclass{standalone}
\usepackage{amsmath}
\usepackage{amsfonts} 
\usepackage{amssymb}
\usepackage{graphicx}
\graphicspath{ {images/} }
\usepackage{xcolor}
%
\pagestyle{empty}
\pagecolor{white}
%
\everymath{\displaystyle}
%
\usepackage{CJKutf8}
\AtBeginDvi{\input{zhwinfonts}}
%\input{zhwinfonts}
\newcommand{\zh}[1]{\begin{CJK*}{UTF8}{zhsong}#1\end{CJK*}}
\newcommand{\YBao}{\color{red}\zh{源宝爱数学}}
\newcommand{\BT}[1]{\color{red}\zh{#1}\color{black}}
\newcommand{\fTxT}[1]{\text{\zh{#1}}}
%

\immediate\write18{pdflatex conic.tex}
\immediate\write18{convert -density 150 -adaptive-resize 480x480 conic.pdf conic.jpg}
%
%\usepackage{pgfplots}
%\pgfplotsset{width=100pt,compat=1.9}
%\tikzset{inner sep=0pt}
% 
\usepackage{multirow}
\newcommand{\LT}[1]{\color{blue}\zh{#1}}
% 
\begin{document}\begin{minipage}[b][14cm][t]{1.00\textwidth}
%
%\begin{table}[htb!]
\begin{minipage}{\linewidth}
\centering
\begin{tabular}{|c|c|c|}
\multicolumn{3}{c}{\Large\BT{园锥曲线:(Conic Section)}} \\[5pt] \hline
\multicolumn{3}{l}{\BT{椭圆:平面内一动点P到两定点F1,F2(焦点)的距离和=定长2a的点集合}} \\ \hline
\LT{标准方程} & \LT{半焦距(c)} & \LT{离心率(e)} \\ \hline
\begin{minipage}{0.5\columnwidth}
  \zh{焦点在x轴上:}$\frac{x^2}{a^2}+\frac{y^2}{b^2}=1\hspace{5pt}(a>b)$\\
  \zh{焦点在y轴上:}$\frac{y^2}{a^2}+\frac{x^2}{b^2}=1\hspace{5pt}(a>b)$
\end{minipage}
& $c^2 = a^2 - b^2$
& $e=\frac{c}{a} \hspace{5pt}{(e < 1)}$ \\ \hline
%
\multicolumn{3}{c}{\BT{双曲线:平面内一动点P到两定点F1,F2(焦点)的距离差=定长2a的点集合}} \\ \hline
\LT{标准方程} & \LT{半焦距(c)} & \LT{离心率(e)} \\ \hline
\begin{minipage}{0.5\columnwidth}
  \zh{焦点在x轴上:}$\frac{x^2}{a^2}-\frac{y^2}{b^2}=1$ \\
  \zh{焦点在y轴上:}$\frac{y^2}{a^2}-\frac{x^2}{b^2}=1$
\end{minipage}
& $c^2 = a^2 + b^2$
& $e=\frac{c}{a} \hspace{5pt}{(e > 1)}$ \\ \hline
% 
\multicolumn{3}{c}{\BT{抛物线:平面内一动点P到一定点F与一条定直线(准线)的距离之比=1的点集合}} \\ \hline
\LT{标准方程} & \LT{半焦距(c)} & \LT{离心率(e)} \\ \hline 
\begin{minipage}{0.5\columnwidth}
  \zh{焦点在x轴上:}$y^2 = \pm 2px$ \\
  \zh{焦点在y轴上:}$x^2 = \pm 2py$
\end{minipage}
& $c = |\frac{2p}{4}|$
& $e = 1$ \\ \hline
\end{tabular}\end{minipage} \\[7pt]
%
\begin{minipage}{\linewidth}
\centering
\begin{tabular}{ll}
\multicolumn{2}{c}{\Large\BT{功德圆满,心想事成}} \\ \hline
\includegraphics{conic_1.jpg}
&
\includegraphics{ybao30.jpg}
\end{tabular}\end{minipage}
%%%
\end{minipage}\end{document}
