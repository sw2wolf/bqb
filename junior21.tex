%pdflatex -shell-escape *.tex
\documentclass{standalone}
\usepackage{amsmath,amsfonts,amssymb}
%
\usepackage{graphicx}
\graphicspath{ {images/} }
%
\usepackage{xcolor}
\pagestyle{empty}
\pagecolor{white}
%
\everymath{\displaystyle}
%
\usepackage{CJKutf8}   %CJKspace
\AtBeginDvi{\input{zhwinfonts}}
%%%
\newcommand{\zh}[1]{\begin{CJK*}{UTF8}{zhsong}#1\end{CJK*}}
\newcommand{\YBao}{\color{red}\zh{源宝爱数学}}
\newcommand{\BT}[1]{\color{red}\zh{#1}\color{black}}
\newcommand{\fTxT}[1]{\text{\zh{#1}}}
%

\immediate\write18{pdflatex junior21.tex}
\immediate\write18{convert -density 200 -adaptive-resize 480x480 junior21.pdf junior21.jpg}
%
\newcommand{\fTxT}[1]{\text{\zh{#1}}}
%
\begin{document}\begin{minipage}[b][14cm][t]{\textwidth}
\begin{center}\large\BT{高二下学期末数学考式21题}\end{center}
\begin{large}
\zh{已知椭圆C:$\frac{x^2}{a^2}+\frac{y^2}{b^2}=1\hspace{3pt}(a>b>0)$的离心率为$\frac{\sqrt{2}}{2}$,点$P(1,\frac{\sqrt{2}}{2})$在椭圆C上,直线l过椭圆的右焦点F且与椭圆相交于A,B两点}   \\
\zh{(1)求椭圆C的方程;}\\
\zh{(2)在x轴上是否存在定点M,使得$\vec{MA}\dot\vec{MB}$为定值? 若存在,求出定点M的坐标;若不存在,请说明理由.}
\end{large}\\[10pt]
%%%
\begin{large}(1)\zh{解:}$\because$ $\left.\begin{aligned}
  e=\frac{c}{a}=\frac{\sqrt{2}}{2} \Rightarrow c^2=\frac{a^2}{2}\\
  c^2=a^2-b^2
\end{aligned}\right\} \Longrightarrow b^2=\frac{a^2}{2} $ \\
\zh{将P代入C} $\Rightarrow \frac{1}{a^2}+ \frac{1}{2b^2}=1$ \\
$\therefore a=\sqrt{2},b=1 \Rightarrow C:\frac{x^2}{2}+y^2=1$
\end{large}\\[10pt]
%%%
\begin{large}(2)\zh{解:}
\end{large}
%%%
\end{minipage}\end{document}
