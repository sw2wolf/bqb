%pdflatex -shell-escape *.tex
\documentclass{standalone}
\usepackage{amsmath,amsfonts,amssymb}
%
\usepackage{graphicx}
\graphicspath{ {images/} }
%
\usepackage{xcolor}
\pagestyle{empty}
\pagecolor{white}
%
\everymath{\displaystyle}
%
\usepackage{CJKutf8}   %CJKspace
\AtBeginDvi{\input{zhwinfonts}}
%%%
\newcommand{\zh}[1]{\begin{CJK*}{UTF8}{zhsong}#1\end{CJK*}}
\newcommand{\YBao}{\color{red}\zh{源宝爱数学}}
\newcommand{\BT}[1]{\color{red}\zh{#1}\color{black}}
\newcommand{\fTxT}[1]{\text{\zh{#1}}}
%

\immediate\write18{pdflatex induct.tex}
\immediate\write18{convert -density 150 -adaptive-resize 480x480 induct.pdf induct.jpg}
%
\begin{document}\begin{minipage}[b][14cm][t]{\textwidth}
\begin{center}\large\BT{高中数学:数学归纳法}\end{center}
\begin{large}
\zh{数学归纳法\textit{(mathematical induction)}是一种数学证明方法,常用于证明命题在自然数范围内成立。虽然数学归纳法名字中有“归纳”,但是数学归纳法并非不严谨的归纳推理法,它属于完全严谨的演绎推理法。}\\[5pt]
\zh{如果想要证明某个命题对于\textbf{所有自然数n}都成立,那么:}\\[3pt]
\zh{一: 证明命题对于n = 1成立}\\
\zh{二: 假设命题对于任意自然数m成立,证明在此假设下,命题对于m+1成立}\\
\zh{三: 命题得证。$\Longrightarrow$} \\[7pt]
%
\color{blue}\zh{n=1成立\hspace{5pt}$\Rightarrow$\hspace{5pt}n=2成立\hspace{5pt}$\cdots$\hspace{5pt}n=m成立\hspace{5pt}$\Rightarrow$\hspace{5pt}n=m+1成立\hspace{5pt}$\cdots$}\color{black} \\[7pt]
%
\zh{这就好像\textbf{多米诺骨牌},当确定第n块骨牌的倒下会导致第n+1块骨牌的倒下后,推倒第一块骨牌,就能保证所有骨牌的倒下。}
\end{large}
%%%
\end{minipage}\end{document}
