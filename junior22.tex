%pdflatex -shell-escape *.tex
\documentclass{standalone}
\usepackage{amsmath,amsfonts,amssymb}
%
\usepackage{graphicx}
\graphicspath{ {images/} }
%
\usepackage{xcolor}
\pagestyle{empty}
\pagecolor{white}
%
\everymath{\displaystyle}
%
\usepackage{CJKutf8}   %CJKspace
\AtBeginDvi{\input{zhwinfonts}}
%%%
\newcommand{\zh}[1]{\begin{CJK*}{UTF8}{zhsong}#1\end{CJK*}}
\newcommand{\YBao}{\color{red}\zh{源宝爱数学}}
\newcommand{\BT}[1]{\color{red}\zh{#1}\color{black}}
\newcommand{\fTxT}[1]{\text{\zh{#1}}}
%

\immediate\write18{pdflatex junior22.tex}
\immediate\write18{convert -density 200 -adaptive-resize 480x480 junior22.pdf junior22.jpg}
%
\newcommand{\fTxT}[1]{\text{\zh{#1}}}
%
\begin{document}\begin{minipage}[b][14cm][t]{\textwidth}
\begin{center}\large\BT{高二下学期末数学考式22题}\end{center}
\color{black}\begin{large}
\zh{已知函数$f(x)=\ln x + \frac{1}{x}+ax(a \ge 0)$}\\
\zh{(1)当a=0时,求f(x)的单调区间;}\\
\zh{(2)当$a>0$时,求f(x)的最小值的取值集合}
\end{large}\\[10pt]
%%%
\begin{large}(1)\zh{解:}
$\because a=0 \therefore f(x)=\ln x + \frac{1}{x}\hspace{3pt}(x > 0)$\\
$\Rightarrow f'(x)=\frac{1}{x}-\frac{1}{x^2}=\frac{x-1}{x^2}$\\
\fTxT{令}
$f'(x)>0 \Rightarrow x>1,f(x)\uparrow$\\
$f'(x)<0 \Rightarrow x<1,f(x)\downarrow$\\
$\therefore$
$x \in (0,1),f(x)\downarrow\hspace{1cm}$
$x \in (1,\infty),f(x)\uparrow$
\end{large}\\[10pt]
%%% 
\begin{large}(2)\zh{解:}
$\because f'(x)=\frac{1}{x}-\frac{1}{x^2}+a=\frac{ax^2+x-1}{x^2}$\\
\fTxT{令}
$f'(x)=0 \Rightarrow ax^2+x-1=0\hspace{5pt}(a>0)$\\
\fTxT{设}$g(x)=ax^2+x-1$\\
$\Rightarrow \Delta=1+4a > 0$\\
$\Rightarrow \fTxT{g(x)零点:}\frac{-1 \pm \sqrt{1+4a}}{2a}$\\
$\because x>0 \therefore \fTxT{g(x)零点只可能}=\frac{-1 + \sqrt{1+4a}}{2a}$\\
$\because x>\frac{-1 + \sqrt{1+4a}}{2a} \Rightarrow g(x)>0 \Rightarrow f(x) \uparrow$\\
$0<x<\frac{-1 + \sqrt{1+4a}}{2a} \Rightarrow g(x)<0 \Rightarrow f(x) \downarrow$\\
$\therefore \fTxT{f(x)的最小值的取值} = \frac{-1 + \sqrt{1+4a}}{2a}$
\end{large}
% 
% \begin{minipage}{\linewidth}
% \centering
% \begin{tabular}{ll}
%   \multicolumn{1}{c}{\Large\BT{治大国若烹小鲜,复杂的事情简单做}} &
%   \includegraphics{ybao30.jpg} \\ \hline
% \includegraphics[scale=0.6]{log_1.png} &
% \includegraphics[scale=0.8]{log_2.png}
% \end{tabular}\end{minipage}
%%%
\end{minipage}\end{document}
